
%---Define your document class, packages, and new commands

\documentclass[11pt]{article}
\usepackage[top=2.54cm, left=2.54cm, right=2.54cm, bottom=2.54cm]{geometry}

\usepackage{amsfonts}   % for statements to be commented out
\usepackage{amssymb}
\usepackage{amsmath}    % for subequations
\usepackage{setspace}
\usepackage{graphicx}


%------------Document-Begins-Here----------------------%

\begin{document}


\title{\textbf{Orbital Patterns of Martian Moons}}

\author{Jake Mathews
\thanks{mathewsj2@wit.edu}\\
Computer Science Undergraduate Program, Wentworth Institute of Technology\\
550 Huntington Avenue, Boston, MA 02115\\
\and Ian Brown
\thanks{browni1@wit.edu}\\
INSERT MAJOR HERE, Wentworth Institute of Technology\\
550 Huntington Avenue, Boston, MA 02115}

\maketitle

\begin{abstract}

This project's goal was to to create a top-down 2-dimentional model
of the positional path of the moons of Mars, Phobos and Deimos.
We created a Fortran 90 program to simulate the orbit of the two moons
and generate data files for later analysis.
The simulation will be provided some initial conditions of a given moon,
and simulate the orbit over the course of one orbital period, as defined
by NASA~\cite{nasa}

\end{abstract}


\section{Introduction}
\noindent Write your introduction here, include pictures, tables, equations, etc...

\section{Theory}
\noindent Do ALL math derivations here...label each equation and reference the
equations accordingly... For example:  From Eq. (2), we see that Eqs. (3) - (7) 
can be simplified, blah, blah...\\

\noindent For every equation, you need to explain each variable, for example:

\begin{equation}
\label{eq:force}
F=\dfrac{mv^2}{r}~.
\end{equation}
\noindent In Eq.~\eqref{eq:force}, $F$ is the force measured in Newtons, $m$ is the
mass in kilograms, $v$ is the velocity measured in meters per second, and $r$ is
the radius of the curved path.  Equation~\eqref{eq:force} was obtain from~\cite{uni}.


\section{Computational Methods \& Techniques}
\noindent Include snipits of your code, DO NOT INCLUDE YOUR ENTIRE CODE HERE!!!!  
Write about the methods you used, make sure you \textbf{\textit{explain}} the methods!!  Don't
just say we used ``RK4", you need to explain what is RK4.

\section{Results}
\noindent Include ALL results here, including tables of results, plots of results, numerical values, etc...
Make sure you include a figure caption for EACH figure.  Make sure you include a table caption for each table.
For example:
\begin{figure}[ht]
\centering
\includegraphics[width=0.5\textwidth, angle =-90]{../presentation/orbits}
\caption{Orbital period in the $x-y$ plane for two full orbits of the Earth orbiting the Sun.}
\label{fig:graph}
\end{figure}
\noindent In Fig.~\ref{fig:graph}, the orbit is set to 2 full periods...

\section{Conclusions}
\noindent Summarize your results...This should be about a 1 page minimum!!!


\begin{thebibliography}{99}

\bibitem{uni} 
Douglas C. Giancoli
\textit{Physics for Scientists and Engineers}. 
Pearson Education Inc., Upper Saddle River, New Jersey, 2009.

\bibitem{nasa} 
NASA
\textit{Mars Fact Sheet}. 
NASA, 2016
nssdc.gsfc.nasa.gov/planetary/factsheet/marsfact.html

\end{thebibliography}





\end{document}